\section{Przykłady}

\textbf{Jak załączyć snippet kodu?}
\begin{lstlisting}[language=Kotlin]
open fun testFunction() {
    isEnabled = true
}
\end{lstlisting}

\vspace{3em}

\textbf{Jak dodać odnośnik do innej części dokumentu?}\\
Patrz dział \hyperlink{targetName}{bibliografii}.
\hypertarget{targetName}{}

\vspace{3em}

\textbf{Jak dodać cytat z bibliografii?}\\
Często korzystam z wyszukiwarki. (google - \cite{google})

\vspace{3em}

\textbf{Jak oznaczyć tekst?}\\
\tcbox{topic}

\vspace{3em}

\textbf{Jak dodać odstęp?}\\

Ho
\hspace{3em}
ryzontalny

Wer\\
\vspace{3em}\\
tykalny

\vspace{3em}

\textbf{Jak działy?}\\

\section{Dział}
\subsection{Poddział}
\subsubsection{Podpoddział}

\newpage

\textbf{Jak dodać rysunek i referencję do niego?}\\

\begin{figure}[h]
    \centering
    \includegraphics[scale=1.005]{pjatk}
    \caption{Logo uczelni}
    \label{fig:customLabel}
\end{figure}

Na rysunku \ref{fig:customLabel} przedstawiono monochromatyczne logo uczelni.

\vspace{3em}

\textbf{Jak dodać obraz szerokości połowy strony z predefiniowanego katalogu?}\\
\vspace{1em}
\includegraphics[width=0.5\textwidth]{pjatk}

\vspace{3em}

\textbf{Jak dodać obraz o określonej szerokości z dowolnego katalogu?}\\
\vspace{1em}
\includegraphics[width=20em]{./images/white/pjatk.png}